\documentclass[a4paper]{standalone}

\pagestyle{empty}

\usepackage{../tikz-multinets}

% Pre-define some cells that can be reused just adding the missing information
% - The first one defines a cell with a shape, label and number of ports.
%   When used, it suffices to add a position and/or direction. 
% - The other three ones define a cell not specifying its name.
%   When used, one add a name and/or position and direction.

\newcommand{\gammacell}
	{\inetmulticelltype[isosceles triangle apex angle=75]{$\gamma$}{1}{2}}
	
\newcommand{\selector}
	{\inetmulticellshape[very thick,inner sep=0pt,isosceles triangle apex angle=55]{1}{3}}
\newcommand{\multiplexor}
	{\inetmulticellshape[very thick,inner sep=-3pt,isosceles triangle apex angle=90]{1}{3}}
\newcommand{\decoder}
	{\inetmulticellshape[very thick,inner sep=0pt,isosceles triangle apex angle=55]{1}{1}}


\begin{document}

\begin{tikzpicture}[show nodes=false]

 \inetmulticell[minimum width=20ex](alpha){$\alpha_i$}{3}{7}{0,0}
 \inetport(alpha.pal 1)
 \inetdotsabove(alpha.pal 2)
 \inetport(alpha.pal 3)
 \inetdotsabove(alpha.pax 4)
% 
 \gammacell(gamma1){alpha.pax 7}++1
 \inetwire(gamma1.pal)(alpha.pax 7)
 \gammacell(gammaN){alpha.pax 2}++1
 \inetwire(gammaN.pal)(alpha.pax 2)
% 
 \selector(Si1){$S_{i:1}$}{gamma1.pax 1}++{1.7}
 \inetwire(Si1.pal)(gamma1.pax 1)
%
 \selector(SiN){$S_{i:n}$}{gammaN.pax 1}++{1.7}
 \inetwire(SiN.pal)(gammaN.pax 1)
%
 \inetmulticell(rho){$\rho_n$}{1}{3}{0.08,4.5}[U]
 \inetshortwire(rho.pax 1)(Si1.pax 2)
 \inetshortwire(rho.pax 3)(SiN.pax 2)
 \inetdotsabove(rho.pax 2)
%
 \multiplexor[arity=7](multiplex){$T_{n+m}$}{rho.pal 1}++{1}
 \inetwirecoords(multiplex.pal)(rho.pal)
 \inetwirefree(multiplex.pax 6)
 \inetabove(multiplex.pax 5){...}
 \inetwirefree(multiplex.pax 4)
 \inetabove(multiplex.pax 2){...}

 \inetnode(n1){gammaN}+{1.2,0}
 \inetnode(nN){n1.e}++{.8}

 \inetnode(m1){n1 |- multiplex.above above pax}
 \inetnode(mN){nN |-  multiplex.above above pax}+{0,.2}
 \inetnode(mN'){mN -| multiplex.pax 3}

 {\inetwiresunder 
  \inetwire(alpha.pax 6)<n1.s><m1.n>(multiplex.pax 1);
  \inetwire(alpha.pax 1)<nN.s><mN.s><mN'>(multiplex.pax 3);

  \selector(Si1'){$S_{i:1}$}{-3.4,2.9}[U]
  \inetshortwire(Si1'.pax)(gamma1.pax 2)
  \selector(SiN'){$S_{i:n}$}{Si1'}+{1.4,0}[U]
  \inetshortwire(SiN'.pax)(gammaN.pax 2)
  \node at ($(Si1')!.5!(SiN') + (0,.2)$) (mid1) {\ldots};
 }
%
 \decoder(D1){$D$}{Si1'.pal}++{.6}[U]
 \inetwirefree(D1.pax)
 \decoder(DN){$D$}{SiN' |- D1}[U]
 \inetwirefree(DN.pax)
%
 \multiplexor[inner sep=1pt](multiplex'){$C_{n}$}%
					{mid1 |- multiplex}[U]
%
 \inetveryshortwire(D1.pal)(multiplex'.pax 1)
 \inetdotsabove(multiplex'.pax 2)
 \inetveryshortwire(DN.pal)(multiplex'.pax 3)
%
 \inetcell[shape border uses incircle=false](copy){$C_2$}%
					{$(multiplex'.pal)!.5!(multiplex.pax 7) + (0,.6)$}[U]

 \inetveryshortwire(multiplex.pax 7)(copy.right pax)
 \inetshortwire(multiplex'.pal)(copy.left pax)
%
 \inetwirefree(copy.pal)

\end{tikzpicture}

\end{document}

% Image magick commands to transform in png or jpg:
%
%%%% Jpeg
%
% 	magick -density 300 -quality 90 MessagePicker.pdf MessagePicker.jpg
%
%%%% Transparent png
%
%	magick -density 300 -quality 90 MessagePicker.pdf MessagePicker.png
%
%%%% Non-transparent png
%
%	magick -density 300 -quality 90 MessagePicker.pdf -background white -alpha remove  MessagePicker.png
