% Copyright 2017 by Sallaccio
%
% This file may be distributed and/or modified
%
% 1. under the LaTeX Project Public License and/or
% 2. under the GNU Public License.
%


%**************************************%
%                NAMES                 %
%______________________________________%

\newcommand{\thisPackage}{Tikz-MultInets\xspace}
\newcommand{\inetPackageLong}{v0.1 tikz interaction nets library by Marc de Falco\xspace}


%**************************************%
%         STYLING FOR COMMANDS         %
%______________________________________%

\newcommand{\command}[1]{/\/\texttt{#1}}
\newcommand{\metablue}[1]{\textbf{\color{blue}\meta{#1}}}
\newcommand{\metagreen}[1]{\textbf{\color{green}\meta{#1}}}
\newcommand{\metared}[1]{\textbf{\color{red}\meta{#1}}}
\newcommand{\metaorange}[1]{\textbf{\color{orange!80!red}\meta{#1}}}
\newcommand{\metagray}[1]{\textbf{\color{gray}\meta{#1}}}
\newcommand{\metabold}[1]{\textbf{\meta{#1}}}


%**************************************%
%               LENGTHS                %
%______________________________________%

\setlength{\parindent}{10pt}
\newcommand{\optionskip}{\hskip-10pt}

%**************************************%
%                BOXES                 %
%______________________________________%

\newtcolorbox{exampleBox}[1][]
{
  %boxrule=0pt,
  halign=flush center,
  colback  = white!95!gray,
  #1,
}

%**************************************%
%          XPARSE COMMANDS             %
%______________________________________%

% Make a colorBox with the code above and the tikz-picture below.
% Needs the use of xparse code to get the argument as verbatim for printing, 
% and rescanning it for use in tikzpicture. 
\ExplSyntaxOn
% "v" means "verbatim argument"
\NewDocumentCommand{\exampleNet}{v}
{
	% store the argument in variable
  	% xparse has absorbed them "verbatim"
  	\tl_set:Nn \l_argument_i_tl { #1 }
  	\begin{exampleBox}
		% print the first argument  
  		\texttt{\l_argument_i_tl}
  	\tcblower
  		% transform back the first argument from verbatim into "standard tokens"
  		\tl_set_rescan:NnV \l_argument_tl {} \l_argument_i_tl
  		% process the contents of the rescanned arguments
  		\begin{center}
  			\begin{tikzpicture}
  				\l_argument_tl
  			\end{tikzpicture}
  		\end{center}
  	\end{exampleBox}
}

% allocate the variables
\tl_new:N \l_argument_tl
\tl_new:N \l_argument_i_tl
% generate a variant command
\cs_generate_variant:Nn \tl_set_rescan:Nnn { NnV }
\ExplSyntaxOff

% Close to above, but two arguments are given: one to print and one to execute inside a tikzpicture.
% In case extra work needs to be done in tikzpicture
\DeclareDocumentCommand\exampleNetCheat{v m}{\begin{exampleBox} \texttt{#1} \tcblower \begin{center}\begin{tikzpicture}[show nodes]  #2 \end{tikzpicture}\end{center}\end{exampleBox}}